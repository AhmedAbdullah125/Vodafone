\documentclass{book}

\begin{document}

\title{Vodafone}
\author{Ahmed Abdullah Nabeeh}
\date{\today}
\maketitle

\tableofcontents
\newpage

\chapter {Business Architecture}
\section {Mission}
Connecting and empowering people and communities, accelerating the development of Egypt.

\section{Vision}
We will be the communications leader in an increasingly connected world.

\section{Challenges}
Vodafone is a large business operating across multiple continents,We have hundreds of millions of customers literally using our mobile phone services and our 
fixed enterprise services and our main challenge today is hou do we delight and keep those customers that are becoming more demanding they're expecting new services
all the time how do innovate at the same pace that they expect us to.

\section{Mergers \& Acquisitions}

\textbf{Vodafone merge with Pepsi.}\\
Every one of them help the himself one by helping the other one, If you drink Pepsi can, you will have free minutes.
And if you use Vodafone ,you will win a Pepsi can.\\
\textbf{Acquisitions}\\
Vodafone Greece completes the acquisition of 72.7% of Hellas Online
Vodafone Group Plc today confirms that Vodafone Greece has completed the 
acquisition of 72.7% of the share capital of Hellas Online SA (“HOL”)
, a leading provider of broadband and fixed-line telephony in Greece, 
for a total cash consideration of €72.7m. Vodafone Greece now owns 91.2% 
of HOL and will extend a mandatory takeover offer for the remaining shares 
in HOL.  The transaction values the fully diluted equity of HOL at €100m and 
is equivalent to an enterprise value of €311m including HOL’s adjusted net 
debt of €211m1.

\section{Future Growth}
Going forward, data is an integral part of our growth story. New technologies such as
fourth-generation (4G) services will provide superior data experience, Vodafone top executive said. 
\\The company is heavily banking and investing in high-speed 4G networks, as a part of data-centric 
business strategy to counter Bharti Airtel's data plans and Indian billionaire Mukesh Ambani's Reliance Jio Infocomm 4G debut this year. 

\chapter {Information Architecture}

\section{Nature Of Vodafone Information}

\begin{itemize}
	
	\item Vodafone users personal information
	\item Internet information 
	\item GPS servics information 
	\item Satellite information  

\end{itemize}

\section{Information User Need To Deal With}
\begin{itemize}
\item The full name
\item National ID
\end{itemize}

\section{Information we need to store}

\begin{itemize}
	\item Vodafone users personal information
\item Vodafone users personal Data (usage , Balance , ... )
\item The phone numbers that you call or send messages	
\item Your transaction history for the purpose or billing and charging.
\item Your account information
\item The date, time and length of the calls and messages you send or receive through our network
\item The level of service you receive
\item The date, time and length of your Internet browsing, and your approximate location at the time of browsing.
\end{itemize}
\section{ Information Growth Rate}
In t beginning of deal between the user and Vodafone ,Vodafone only have your main personal information.
By the time, Vodafone can collect more information from your using as (phone numbers that you call, transaction history. account information, The date, time and length of the calls, messages, Internet browsing)
\section{Best / Most suitable storage solution}
The most suitable solution to store enormous data for millions of customers is \textbf{Cloud Computing}.
To save the data for as long as needed.
	 



\section{How Information Architecture Will Affect Our Future Capabilities}
In Vodafone information is growing very fast,so will need to increase the storage capacity, but yet we need to keep the old information save as we might need it in the future. 
\chapter{Information Technology Architecture}
\section{What Is Our IT Infrastructure Capabilities?}
\begin{itemize}
	\item Network infrastructure Connects all customers together and enables the Group to provide mobile and fixed voice, messaging and data services.	Vodafone operates 2G networks in all of its mobile operating subsidiaries and an increasing number of 3G networks, providing customers with an enhanced data experience. Vodafone also operates an increasing number of fixed access networks.
\item Network infrastructure Vodafone’s network infrastructure provides the means of delivering the Group’s mobile and fixed voice, messaging and data services to its customers. The Group’s customers are linked via the access part of the network, which connects to the core network that manages the set-up and routing of calls, transfer of messages and data connections, which provide a wide variety of other services.
\end{itemize}
\section{Does It Fit Our Needs?}
With these information technologies system, IT infrastructure that Vodafone uses ,Vodafone fits to its needs.And Vodafone Also work on its infrastructure to 	more fit its needs.

\section{Does it fit our future needs?}
In communication world , the world in evolving extremely fast.
If Vodafone still using the current infrastructure, it will not be able to keep up with other competitors and it will lose its customers loyalty.


\section{How Can We Face The Information Architecture Challenges}

Vodafone has attracted millions of customers around the world. It's becoming harder to store everyone's data as it is growing very fast and unpredictable. they could overestimate and overpay for extra capacity storage using a traditional storage.

with cloud storage they can overcome this problem.

\section{Shall We Move To Cloud Computing?}
Vodafone already started to use cloud computing

\section{Which Model We Use?}
Vodafone use Hybrid model in its cloud computing, so Vodafone Employees can manage customers' information, without exposing the customers' privacy.

\section{What are the criteria to focus on when choosing between different cloud computing providers?}
\subsection{Know Your Business Objectives}
Before signing with a cloud provider, make sure that provider is fully committed to understanding your business and the specific objectives you hope to achieve with cloud, said Puneet Shivam, head of U.S. and global co-head of the outsourcing vertical at Avendus Capital, Inc., a financial services firm based in New York

\subsection{Security and cost}
The location of a cloud provider's data centers should also be considered, said Catherine Spence, principal engineer and cloud architect at Intel and chair of the Technical Coordination Committee for the Open Data Center Alliance (ODCA).

\subsection{Data management}
You may already have a data classification scheme in place that defines types of data according to sensitivity and/or policies on data residency. At the very least you should be aware of regulatory or data privacy rules governing personal data.

With that in mind, the location your data resides in, and the subsequent local laws it is subject to, may be a key part of the selection process. If you have specific requirements and obligations, you should look for providers that give you choice and control regarding the jurisdiction in which your data is stored, processed and managed. Cloud service providers should be transparent about their data center locations but you should also take responsibility for finding this information out.  
\chapter{Software Architecture}
\section{What is the best software architecture for Vodafone?}
Vodafone is a large company .it can be using any software architecture such as (client server, N-tier, Peer-to-peer) or even an architecture we have not even heard of. Or they can use a combination of all .
I suggest they can build a new architecture of their own.

\section{How this software architecture helps with facing the mentioned challenges?}

If Vodafone built its own software architecture, It should be very customize to suit its needs and It can face any facing challenges. 	




\end{document}